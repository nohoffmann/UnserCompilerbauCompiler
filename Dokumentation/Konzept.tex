% !TEX root = Main.tex

\section{Konzeption der Entwicklung}
\subsection{Allgemeiner Grundsatz}
\label{sec:grundsatz}
Während der Entwicklung versuchten wir uns möglichst kleine Zwischenziele zu setzen, um die Aufgabe in überschaubare, einfach zu lösende Teilprobleme zu unterteilen. Rückblickend war dies eine sinnvolle Strategie, die zum gewünschten Ziel führte.


\subsection{Wahl der Entwicklungswerkzeuge}
Bezüglich der Entwicklungswerkzeuge wurden uns verschiedene Möglichkeiten vorgestellt. Die erste Möglichkeit bestand darin das Tool ''lex'' beziehungsweise dessen Open-Source-Implementierung ''flex'' als Lexer-Generator und das Tool ''yacc'' beziehungsweise dessen Open-Source-Implementierung ''bison'' als Parser-Generator zu nutzen. Die zweite Möglichkeit bestand in der Nutzung des Werkzeuges ANTLR, das die Funktionalitäten der zuvor genannten Werkzeuge in einem Programm zusammenfasst.

Da ANTLR automatisiert auf Grundlage einer Datei, die die von uns formulierte Grammatik der Sprache enthält, Lexer sowie Parser zeitgleich erstellen kann, entschieden wir uns für diese Variante. Der Vorteil besteht dabei darin, dass zwei essentielle Teile des Compilers von einem Tool übernommen werden. 

\subsection{Herkunft des Namens UnsereCompilerbauSprache}
Für die Programmiersprache wurde ursprünglich der Name C= gewählt (gesprochen: C equal). In einem vorherigen Schritt sollte die Programmiersprache C - - genannt werden, in Anlehnung an den verkleinerten Sprachumfang der Sprache C sowie der Namensgebung von C++. Da C - - jedoch bereits als Zwischensprache existiert, wurde die im Rahmen des Moduls Compilerbau entwickelte Sprache in Anlehnung an C\# (\# ist auch als vier aneinandergereihte Additionssymbole zu interpretieren) zunächst C= genannt, wobei das = vier aneinandergereihte Subtraktionszeichen darstellen soll. Diese Idee wurde jedoch im Laufe der Projektarbeit verworfen und für den Namen der Programmiersprache wurde sich auf \textit{UnsereCompilerbauSprache} geeinigt, um einen etwas besser zum Projekt passenden Namen zu verwenden

\subsection{Sprachdesign}
Es wurde sich darauf geeinigt, dass die Syntax der von uns entworfenen Ausgangssprache der Syntax der Programmiersprache C ähnlich sein soll, da dies gleich mehrere Vorteile hervorbringt:
\begin{itemize}
\item Alle Projektteilnehmer sind mit der Sprache C vertraut
\item C ist eine der am weitesten verbreiteten compilierten Sprachen
\item Die Sprache C besitzt aufgrund seiner Klammerregeln eine gute Lesbarkeit
\end{itemize}

\subsection{Implementierung}
Unser Compiler verwendet im Zusammenspiel mit dem von ANTLR generierten Lexer und Parser einen Visitor für die syntaxgesteuerte Übersetzung. Die Alternative dazu besteht in einem Listener, der sich von einem Visitor dahingehend unterscheidet, dass die verschiedenen Methoden die auf die Knoten des Syntaxbaumes angewandt werden, keinen Rückgabewert besitzen und die Ergebnisse dieser Methoden somit separat gespeichert werden müssen. Der Vorteil eines Listeners hingegen besteht darin, dass untergeordnete Knoten im Syntaxbaum automatisch bearbeitet werden und nicht wie beim Visitor explizit besucht werden müssen. Nach der Abwägung dieser Vor- und Nachteile entschieden wir uns für einen Visitor, da dieser uns als die einfachere Methode erschien.



