% !TEX root = Main.tex

\section{Anforderungen}

Um zu gewährleisten, dass alle Modulteilnehmer über die gleichen Voraussetzungen verfügen und nach festgelegten Kriterien bewertet werden können, wurden durch den Modulverantwortlichen einige Rahmenbedingungen vorgegeben. Die Prüfleistung besteht dabei aus der Entwicklung einer einfachen Programmiersprache, dem Anfertigen einer Dokumentation sowie der Präsentation der Ergebnisse. Die Anforderungen an das Projekt werden in den folgenden Abschnitten aufgeführt und erläutert.

\subsection{Infrastruktur}
Das Ziel ist die Enwicklung eines Frontends, welches Quelltext in der von uns entwickelten Programmiersprache zu Jasmin-Code übersetzt. 
Jasmin ist ein Assembler für die Java Virtual Machine, der Quelltext in einer assembler-artigen Syntax zu Bytecode übersetzt. 
Dieser Bytecode kann schließlich von der Java Virtual Machine ausgeführt werden.


\subsection{Programmiersprache}
Die von uns entwickelte Programmiersprache besitzt folgende Komponenten: 
\begin{itemize}
\item Bezeichner
\item Konstanten
\item Variablen
\item Arithmetik mit Grundrechenarten
\item Klammerung
\item Funktionen
\item if-else-Anweisung
\item while-Schleife
\end{itemize}
\noindent
Die Entwicklung wird mit 50\% in der Gesamtbewertung gewichtet.
In einem späteren Kapitel dieser Dokumentation wird erläutert, wie die Programmiersprache verwendet wird.

\subsection{Dokumentation}
Es ist eine Dokumentation anzufertigen, die Aufschluss über die Konzeption der Entwicklung, die verwendeten Technoligen, das Vorgehen während der Entwicklung, aufgetretene Probleme und Lösungsansätze zu diesen sowie die Verwendung des Compilers gibt.

Die Dokumentation wird mit 30\% in der Gesamtbewertung gewichtet.

\subsection{Präsentation}
In einer circa 20-minütigen Präsentation werden die verwendeten Technologien, wichtige Entwicklungsschritte sowie aufgetretene Probleme erläutert. Außerdem wird der Compiler vorgeführt. 

Die Präsentation wird mit 20\% in der Gesamtbewertung gewichtet.
