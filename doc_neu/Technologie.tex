% !TEX root = Main.tex

\section{Verwendete Technologien}

\subsection{Eclipse}
Zur Entwicklung wurde die integrierte Entwicklungsumgebung Eclipse Photon in der aktuellesten Version (4.8.0) genutzt. 
Eclipse bietet mehrere Vorteile, die ein einfacheres und effizienteres Entwickeln erlauben. Eclipse enthält einen Datei-Explorer, mit dessen Hilfe die Navigation durch die verschiedenen Verzeichnisse zusammen mit Editor und Konsole innerhalb eines Fensters geschehen kann. Des Weiteren bietet Eclipse automatische Vervollständigungsvorschläge an, was die Schreibgeschwindigkeit erhöht. 
Außerdem kann Eclipse durch Plug-Ins erweitert werden. Wir nutzten beispielsweise das Testframework TestNG als Plug-In, um verschiedene Testfälle automatisiert zu prüfen.

\subsection{TestNG}
% TO-DO: Versionsnummer prüfen
TestNG ist ein Framework, das dem zu testenden Programm Eingabewerte übergibt und die tatsächlichen Ergebnisse des Programms mit den erwarteten Ergbnissen abgleicht. Dabei können beliebig viele Testszenarien geprüft werden. Es werden sogenannte positive Tests, also solche, bei denen die Eingabe ein erwartetes Ergebnis hervorruft, als auch negative Tests, bei denen geprüft wird, ob nicht vorhergesehene Eingaben mit entsprechenden Fehlermeldungen korrekt behandelt werden, durchgeführt. TestNG wurde von uns in Version 6.10 (PRÜFEN!) genutzt.
\linebreak
Als Beispiel: Der Quelltext ''println(1+4);'' sollte als Ergebnis ''5'' auf der Konsole ausgeben. TestNG übergibt den Quelltext an unseren Compiler, ruft Jasmin auf, führt das Programm aus und gleicht dann die Ergebnisse ab. Wenn das Ergebnis ''5'' ist gilt der Test als bestanden, andernfalls als durchgefallen. 

\subsection{ANTLR}
ANTLR steht für \textit{ANother Tool for Language Recognition} und ist ein Lexer- und Parsergenerator. ANTLR generiert auf Grundlage einer Grammatik-Datei Java-Code, der einen entsprechenden Lexer sowie ein Template für Teile des Parsers implementiert. Dadurch, dass die von ANTLR generierten Programme aus einer Eingabedatei in der Ausgangssprache einen Syntaxbaum erstellen können, besteht unsere Arbeit größtenteils daraus, Grammatiken für ANTLR zu formulieren und Teile des Parsers (in unserem Fall ein Visitor) zu implementieren. Wir haben die ANTLR-Version 4.7.1 verwendet.

\subsection{Jasmin}
Jasmin ist ein Assembler für die Java Virtual Machine. Dabei werden Textdateien mit Anweisungen in Jasmin-Syntax zu Bytecode übersetzt. Dieser Bytecode kann von der JVM ausgeführt werden.

\subsection{Java}
% TO-DO: Finalen Namen der Sprache einfügen
Die Programmiersprache Java wurde für das Projekt gewählt, da der von ANTLR generierte Code ebenfalls in Java ist. Es ist teilweise notwendig von Klassen des ANTLR generierten Code abzuleiten, weshalb die Wahl einer anderen Programmiersprache keinen Sinn ergeben hätte. 
Des Weiteren muss die Java Runtime Environment ohnehin genutzt werden, um ANTLR, Jasmin sowie schließlich die Kompilate des HIERNAMEUNSERERSPRACHEEINFÜGEN-Compilers auszuführen.

\subsection{Java Virtual Machine}
Die Java VM ist ein Zwischenschritt beim Ausführen von Java-Code. Eine Java-Quelldatei (.java) wird zunächst durch den Java-Compiler zu Bytecode (.class) übersetzt und dann von der Java VM interpretiert. Dieser Zwischenschritt ermöglicht eine Plattformunabhängigkeit, da die Kompilate (.class-Dateien) nicht maschinenspezifisch übersetzt werden. Plattformabhängiger Maschinencode wird erst von der Java VM generiert, weshalb jeder Bytecode ausgeführt werden kann, solange für die entsprechende Maschine die Java VM verfügbar ist.

In diesem Projekt wird die Java VM jedoch nicht nur zur Übersetzung unseres Code genutzt, sondern auch um die Kompilate unseres eigenen Compilers auszuführen. Diese Kompilate werden von Jasmin zu Bytecode übersetzt, der Bytecode wiederum wird von der Java VM ausgeführt.

Zu beachten ist, dass die Java VM stack-basiert operiert und nicht wie beispielsweise der x86-Befehlssatz mit Registern arbeitet, was bei der Implementierung zu beachten ist und eine besondere Denkweise erfordert. 
