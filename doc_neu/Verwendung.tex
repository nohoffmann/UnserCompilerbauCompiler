% !TEX root = Main.tex

\section{Verwendung der UnsereCompilerbauSprache}
Ein gültiges Programm in der UnsereCompilerbauSprache besteht aus beliebig vielen Statements und Funktionsdefinitionen. 

\subsection{Datentypen}
Der einzige Datentyp der Sprache ist 'int'. Ein 'int' repräsentiert eine ganze Zahl mit einer Größe von 32 bit. Mögliche Werte liegen zwischen -2147483648 und 2147483647 ($-2^{31}$ bis $2^{31} - 1$).

\subsection{Bezeichner}
Bezeichner werden genutzt, um Variablen, Konstanten sowie Funktionen zu bennnen.
Beziechner müssen eindeutig sein, d.h. ein Bezeichner darf im entsprechenden Gültigkeitsbereich in nur einer Deklaration vorkommen. Gültige Bezeichner beginnen mit einem Klein- oder Großbuchstaben und dürfen von belibig vielen weiteren Klein- bzw. Großbuchstaben sowie den Ziffern von 0 bis 9 gefolgt werden. Ein Bezeichner darf kein reserviertes Schlüsselwort sein (vgl. Liste der reservierten Schlüsselwörter).

\subsection{Numerische Werte}
Numerische Werte bestehen aus einer beliebig langen Folge de Ziffern von 0 bis 9. Ein numerischer Wert muss aus mindestens einer Ziffer bestehen.

\subsection{Besondere Formatierungszeichen} 
Leerzeichen, Tabstops sowie Zeilenumbrüche dürfen in einer Quelltext-Datei vorkommen, werden jedoch vom Compiler ignoriert.


\subsection{Statements}
Alle Statements ausser if-else-Anweisungen und Schleifen werden mit einem Semikolon '';''.

\subsection*{Variablen}
\subparagraph{Deklaration}
Variablen werden mit dem Schlüsselwort 'int' sowie dem gewünschten Namen nach einem Leerzeichen deklariert.
Der Name wird durch einen Bezeichner angegeben.

Beispiel:
\begin{lstlisting} [frame=single] 
int beispiel;
\end{lstlisting}



\subparagraph{Definition}
Nach dem eine Variable deklariert wurde, kann ihr ein Wert zugewiesen werden. Eine Zuweisung erfolgt durch das Gleichheitszeichen ''=''. Die Linke Seite der Zuwesiung ist dabei der Name der Variable, die rechte Seite  kann ein arithmetischer Ausdruck, der Aufruf einer Funktion, eine Variable oder eine Konstante sein.
Wertzuweisungen dürfen beliebig oft erfolgen.
Die Zuweisung muss getrennt von der Deklaration erfolgen.

Beispiel:
\begin{lstlisting} [frame=single] 
beispiel = 42;
beispiel = eineFunktion();
beispiel = 1 + 2 + 3;
beispiel = eineAndereVariable;
beispiel = EINE_KONSTANTE;
\end{lstlisting}

\subparagraph{Aufruf}
Variablen dürfen auch in der rechten Seite einer Zuweisung vorkommen.
Beispiel:
\begin{lstlisting} [frame=single] 
int beispiel1;
int beispiel2;

beispiel1 = 42;
beispiel2 = beispiel1 - 1;
\end{lstlisting}


	- Variablen deklarieren, definieren, aufrufen
	- Konstanten "	
	- println-Funktion
	- Funktionen deklarieren, aufrufen
	- Arithmetik
	- Aussgaenlogik
	- if-else-statements
	- while-Schleife
	
\section{Aufrufen des UnserCompilerbauCompilers}
	- java -jar UCC.jar sourceCode.txt
  (	- java -jar jasmin.jar output.j)
  	- java output
